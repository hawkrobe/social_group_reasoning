\documentclass[12pt, oneside]{article}   	% use "amsart" instead of "article" for AMSLaTeX format
\usepackage[margin=1in]{geometry}                		% See geometry.pdf to learn the layout options. There are lots.
%\geometry{letterpaper}                   		% ... or a4paper or a5paper or ... 
\usepackage{apacite}
\usepackage{graphicx}				% Use pdf, png, jpg, or eps§ with pdflatex; use eps in DVI mode
								% TeX will automatically convert eps --> pdf in pdflatex		
\usepackage[left]{lineno}
\usepackage{amssymb}

%SetFonts

%SetFonts


\title{Intuitive Theories of Social Groups}
\author{Robert Hawkins}
%\date{}							% Activate to display a given date or no date

\begin{document}
\maketitle
%\subsection{}

\linenumbers

\section*{Introduction}
How did your parents feel on the day you were born? What will the Supreme Court do in that upcoming voting rights case? Why did the US military fail so disastrously in Vietnam? How did Greece respond to the austerity measures imposed by its creditors?

We routinely treat groups of people, even those as large and diverse as the population of Greece, as holistic entities with thoughts, intentions, and feelings distinct from their constituents \cite{BloomVeres99_IntentionalityOfGroups, TheinerAllenGoldstone10_GroupCognition, JenkinsEtAl14_NeuralGroupAgents}. We're comfortable attributing sophisticated properties of mind to these groups \cite{WaytzYoung12_AttributingMindToGroups}, blaming the group as a whole when things go wrong \cite{LangeWashburn12_CorporateIrresponsibility}, and predicting what the group will do in the future. While in some cases it is \emph{possible} to think about the group as the sum of its parts -- you could in principal get a good sense of how your parents felt at your birth by individually asking your mother and father -- this is clearly impossible for larger groups. How, then, do we form stable representations of group-level properties, how do we update our beliefs about these group-level properties from observing individuals, and how do we make decisions on the basis of these beliefs? 

Questions about group structure and the way individuals relate to collectives were critical to early research programs in social psychology. When Lewin \citeyear{Lewin52_GroupDecisionSocialChange} studied the surprising effectiveness of group discussion on housewives' commitment to meat consumption, he pointed out that ``the group level itself acquires values.'' The discussions were effective, in part, because they allowed members to change their beliefs about the standards of the group to which they belong. Asch's \citeyear{Asch51_Conformity} group conformity experiments demonstrated that many individuals weight group-level information, gathered from observing other members of their group, even more strongly than their private perceptual information. If you see yourself as part of a group, and assume that people within a group should be similar, then it's rational to rely on overwhelming group-level evidence. Festinger and colleagues \citeyear{FestingerSchacterBack50_Book} were interested in attributes of groups, such as cohesiveness, which make the group attractive to its members, while Tajifel and colleagues \citeyear{TajfelEtAl71_MinimalGroup} extended this work by considering group \emph{identity} and the way we reason about members of out-groups. 

Research on groups has waned in popularity since the early days of social psychology, however, and many of the recent developments in the social cognition tradition have focused solely on how we reason about other \emph{individuals} \cite{RossLepperWard09_HistorySocialPsych}. Increasingly, there is a call for reintegrating these new cognitive paradigms and insights to reasses classic questions about groups \cite{AbelsonEtAl98_CollectiveOther, SedikidesEtAl98_IntergroupCogntion, LickelHamiltonSherman01_LayTheoryGroups}. 

In this paper, we review the literature on group-level reasoning at the intersection of social psychology and cognitive psychology, attempting to outline an ``intuitive theory of social groups''. We proceed by analogy to the many domains where cognitive scientists have fleshed out ``intuitive theories," such as physics \cite{McCloskey83_IntuitivePhysics, BattagliaHamrickTenenbaum13_SimulationPhysicsPNAS}, biology \cite{Atran98_FolkBiology}, intentionality and animacy \cite{HeiderSimmel44_Animacy}, and category learning more broadly \cite{MurphyMedin85_TheoriesConceptualCoherence, Carey85_ConceptualChange}. In other words, we suppose that all adults hold some basic assumptions about how groups are structured, how traits co-vary between members of a group, how individuals come to belong to a group, and so on. We'll begin by formalizing these assumptions in a simple computational model, and then show how this model formalizes a number of observations from the social psychology literature.

\section*{A simple Bayesian model of intergroup cognition}

To clarify the key components of an intuitive theory of groups, we formulate a simple computational model of a Bayesian agent who reasons about group structure and uses high-level group properties to resolve uncertainty over their own interests. To simplify the explanation, we will consider this agent's behavior in the specific domain of choosing which restaurant to visit. 

Suppose our agent, Alice, has moved to a new town with three restaurants: Burger Barn ($r_1$), Taco Town ($r_2$), and Stirfry Shack ($r_3$). Alice has some true reward function $\rho : r_i \rightarrow [0,1]$ assigning a noisy payoff rate to each restaurant. For instance, if $\rho(r_1) = .5$, then Alice will have a good experience eating at Burger Barn 50\% of the time and a bad experience 50\% of the time. Alice doesn't have access to her true reward function, but when she visits a restaurant $r_i$, she observes a reward signal $s_{r_i}$ sampled from a Bernoulli distribution with probability given by her reward function: $$s_{r_i} \sim \textrm{Bernoulli}(\rho(r_i))$$ Finally, Alice can also observe the choices made by a population of $N$ other agents $\mathcal{A} = \{a_1, \dots, a_N\}$. 

How does Alice infer her reward function? Classic reinforcement learning models have shown that Alice can eventually learn her reward function using only her reward signal as evidence \cite{SuttonBarto12_ReinforcementLearning}, but it would require her to spend a lot of time at restaurants! Additionally, if there were thirty restaurants instead of three, she wouldn't have time to visit them all. Social information can guide her inference, so that she learns more efficiently. 

How does she integrate social information? Intuitively, we'd expect that Alice would prefer to learn from those who have similar reward functions. This is where assumptions about group structure are useful. We formulate Alice's beliefs about the other agents $\mathcal{A}$ as a generative probabilistic model. We describe this model in four steps:
\begin{enumerate}
\item Alice assumes that agents within a group $g_k$ share a common reward function $\rho_k$ with some mean reward rate $\mu_k$ and with some variability $\sigma^2_k$ around that mean. 
\item Alice has uncertainty about the values of these group-level parameters, represented by her prior beliefs: $\mu_k \sim \textrm{Unif}(0,1)$, $\sigma^2_k \sim \textrm{Unif}(0, 0.1)$
\item Alice assumes that each agent $a_j$ belongs to exactly one\footnote{This assumption can be relaxed by allowing an identity distribution over groups for each agent} group, and that their reward function is drawn from a Gaussian distribution with the group mean and variance: $\rho_{a_j \in g_k} \sim \textrm{Gaussian}(\mu_k, \sigma^2_k)$.
\item Alice assumes that other agents know their reward functions and choose which restaurant to visit according to a soft-max decision rule, i.e. $$P(r_j | \rho) = \frac{\rho(r_j)}{\sum_j \rho(r_j)}$$
\end{enumerate}

Through Bayesian inference, Alice can use this intuitive theory to infer the modal reward function for each group ($\mu_k$), and also how similar those within a group are likely to be to one another ($\sigma^2_k$). At the same time, she can infer which group each agent is most likely to belong to, and what her own reward function is likely to be. We implemented this cognitive model (performing inference using Sequential Monte Carlo) in the probabilistic programming language WebPPL \cite{GoodmanStuhlmuller14_DIPPL}, and it is available online for further experimentation\footnote{\url{http://forestdb.org/models/social-utility.html}}.

Note that this is intended to be a conceptual framework that can be easily extended to account for subtler scenarios: for example, Festinger's \citeyear{FestingerSchacterBack50_Book} observations of spatial ecology can be modeled by placing agents on a network structure, such that they can only observe their neighbors \cite{Watts14_SocialNetworksVoting}. Stereotyping can be modeled by giving agents incidental properties such as skin color, which may not reflect their reward function, and allowing these properties to co-vary within groups. Social conventions can be modeled by relaxing the true reward signal to be dependent only on other other agents' choices \cite{Lewis69_Convention}, and shifts in group identity \cite{TajfelEtAl71_MinimalGroup} can be modeled by giving the agent strong evidence of her own reward function but weak evidence for her group membership, or vice versa. 

\section*{Entitativity}

With this conceptual framework in mind, we turn to the social psychology literature. Most recent social psychology studies investigating an intuitive theory of groups focus specifically on a single property of groups: \emph{entitativity}. The term was coined by Donald Campbell \citeyear{Campbell58_AggregatesAsEntities} to refer to the extent to which an aggregate of discrete agents can be perceived as a whole. As an example, he writes, ``a band of Gypsies is empirically harder, more solid, more sharply bound than the ladies aid society, and the high school basketball team during basketball season falls somewhere in between.'' Much later, Hamilton and Sherman \citeyear{HamiltonSherman96_PerceivingPersonsGroups} used this idea in an influential analysis of how people perceive individuals, as opposed to groups: ``Groups vary in one's perception of their entitativity, and therefore so do one's expectations of consistency among their members.'' They pinpointed entitativity as the key property that governs how individuals reason about groups: a tight-knit, high entitativity group like a fraternity is likely to be represented and reasoned about in the same way as an individual (i.e. using the usual theory of mind apparatus), while a looser coalition like the Democratic party is not. Entitativity plays an important role in distinguishing different varieties of groups within an intuitive taxonomy \cite<e.g. intimacy groups vs. social categories;>{LickelEtAl00_GroupEntitativity}

This notion of entitativity is captured by the group variance parameter $\sigma^2_k$ of the computational model outlined above. By observing the actions of individual agents over time, Alice can infer whether the group they belong to is more tight-knit or more dispersed. If she believes a group's variance is high, i.e. that entitativity is low, then evidence showing that some individual belongs to that group is fairly uninformative -- it does not tell her much about that person. If she believes she also belongs to a low entitativity group, then social information from fellow group members is not likely to be useful in learning her own reward function. 

We do not have room here to discuss the full range of experimental studies on entitativity perception and the consequences of entitativity; instead, we focus on a particularly prominent application of the idea to stereotyping. In a study by Dasgupta, Banaji, and Abelson \citeyear{DasguptaBanajiAbelson99_Entitativity}, participants saw collections of novel creatures called Gs\footnote{``Greebles,'' which are famous for their use in object recognition studies}, which either shared a common color (the high entitativity condition) or varied in their color (the low entitativity�condition). One each trial, a picture of a collective or an individual was displayed and participants  were asked to judge the creatures' attitudes and likely behaviors (e.g. ``How threatening are these Gs toward the Hs?'') They found (1) that high entitativity groups (defined in terms of appearance) were also judged to be more psychologically homogeneous, and (2) that high entitativity groups were judged more negatively than low entitativity groups. In other words, low variability in a single trait (color) was extended to other traits and biased attitudes against the group. This is clearly an analogue for stereotyping: when whites see minorities with low variability in some physical feature like skin color, they assume that minority will have low variability in other traits as well, possible in some salient negative trait. We could reproduce this scenario in our model by adding another dimension of decision, and assuming that reward functions on both dimensions are (weakly) correlated. If Alice infers group structure based solely on restaurant choices, and we ask her who she believes is most likely to share her music tastes, then in the absence of further information, she will point to those in her restaurant group.

A recent study by Waytz and Young \citeyear{WaytzYoung12_AttributingMindToGroups} takes the literature on entitativity in a different direction. They're interested in the tradeoff between the traits we attribute to a group and the traits we attribute to its members: if people see a group, like a fraternity, as highly entitative and acting as a cohesive unit, are they less to attribute mind to its members? For instance, when a corporation acts as whole to commit fraud or environmental devastation at a massive scale, we tend to view its members as ``cogs in the machine'' who mindlessly act toward group-level goals. To test this, they asked participants to rate 20 different groups on three factors: the extent to which the group has ``a mind of its own,'' the extent to which the average member of each group has a mind of her own, and the extent to which each group is cohesive. They found (1) a negative correlation between attributions of mind to the group and attributions of mind to its members and (2) a strong positive correlation between cohesion and attributions of group mind. They then applied this to the specific issue of attributing \emph{moral responsibility} to groups vs. their members, a topic with a long history in social psychology \cite{Bandura99_MoralDisengagement, Zimbardo07_Lucifer}. 

It is, however, unclear what participants are doing when asked whether a group has ``a mind of its own,'' especially in the concrete computational terms of our model. Simply asking the question presupposes that groups have minds, which may bias participants to consider actions that can be taken at the group level, as opposed to the individual level. Also, asking about the minds of the individuals \emph{after} asking about group mind may introduce unexpected task demands: it seems to be asking about participants' agency and intentionality \emph{with respect to a group-level action} that was primed by the initial question. One concrete way of reframing this question in our model, then, is to ask how Alice would predict a particular agent's next action: if Alice knows that this agent is part of a highly entitative (low variance) group, then Alice will prefer to use the \emph{group mean} rather than the \emph{agent's} reward function.  In this sense, she has attributed mind (i.e. goals and intention) to the group at the expense of the member. 

\section*{Conclusion}

Groups are a staple of social life: they shape our values, influence our behavior, and organize our identities. Yet groups are also defined by the individuals who belong to them. Resolving these co-dependencies is one of the foundational questions of social psychology. In this paper, we suggested that it may be fruitful to tackle this problem by articulating an intuitive theory of social groups. The set of assumptions forming this intuitive theory govern the way we reason about group-level properties like entitativity, make inferences about individuals given evidence of their group membership, and learn about our own rewards and identities. We formalized a minimal set of such assumptions in a simple computational model, where groups are represented as sets of people with reward functions distributed with some variance around a shared mean. Groups vary in their variances, leading to more or less cohesive collectives. This model can be extended in many ways as a tool to explore different theories, but we saw that it was able to (conceptually) account for well-known results like stronger stereotyping of high entitativity groups \cite{DasguptaBanajiAbelson99_Entitativity} and the tradeoff in using group-level vs. member-level reasoning for more or less entitative groups \cite{WaytzYoung12_AttributingMindToGroups}. 

In addition to basic cognitive questions like how we represent and process information about groups, there are several deeper questions that we believe an intuitive theory of social groups would help address: How do we explain the ``illusion of utility,'' where we value something for many years (e.g. a particular job or hobby or relationship) and then suddenly realize that our value came primarily from our social group instead of the thing itself? How does group membership evolve over multiple generations, giving rise to a ``cultural ratchet'' where agents are able to explore rare but potentially high-value parts of the action space and propagate that information? How do agents cluster themselves into groups in a new environment? How do conventions form within groups? Formalizing our hypotheses about group perception and reasoning in computational models allows us to synthesize previous findings, test concrete, quantitative predictions from competing models, and provide a unified framework for answering these questions.

\bibliographystyle{apacite}
\bibliography{bib}

\end{document}  
